\title{Calculation of collector's curves}
\author{}
\date{}

\documentclass[12pt]{scrartcl}
\usepackage{amsmath,amsthm,amssymb}
\usepackage{hyperref}
\hypersetup{
    colorlinks=true,
    linkcolor=blue,
    filecolor=magenta,      
    urlcolor=cyan,
}
\urlstyle{same}
\begin{document}
\maketitle

\section*{Haplotypes, orthogroups, and scores}

In pangenomics, collector's curves show the relationship of the number of haplotypes (here $H$) to the number of gene families or orthogroups (here $X$).

Given the $X$ orthogroups distributed across $H$ haplotypes, let the \emph{score} $s_x \in [0,H]$ of an orthogroup $x$ be the number of haplotypes in which $x$ is present. For any score $s$ let $P(s)$ be the number of orthogroups with score equal to $s$. 

\begin{align}
    P(s) = \sum_{x \in x_0 ... x_X} I_{s_x=s}(x)
\end{align}

Where $I_{s_x=s} : \{x_0 ... x_X\} \rightarrow \{0,1\}$ is the indicator function on $\{x \in x_0 ... x_X : s_x=s\}$. 

\section*{The collector's curves}

The collector's curve $C(h):[1,H] \rightarrow [0,X]$ is the expected number of orthogroups that will be present in a subset of $h$ haplotypes randomly drawn from the total set of $H$. It can be calculated by:

\begin{align}
    C(h) = \sum_{s \in 1...H} 1 - P(s)\prod_{i \in 0...h-1}\frac{H-s-i}{H-i}
\end{align}

The expected number of \emph{core} orthogroups $\hat{C}(h)$ can be estimated by

\begin{align}
    \hat{C}(h) = \sum_{s \in 1...H} P(s)\prod_{i \in 0...h-1}\frac{s-i}{H-i}
\end{align}

Each of these  is a special case of a general formula for the expected number of orthogroups with a score of at least $n$, based on the hypergeometric survival function:

\begin{align}
    C_n(h) = \sum_{s \in 1...H}P(s)S_\text{hyp}(n, H, s, h)
\end{align}

Where $S_\text{hyp}(n, H, s, h)$ is the \href{https://docs.scipy.org/doc/scipy/reference/generated/scipy.stats.hypergeom.html}{hypergeometric survival function} or the hypergeometric cumulative distribution function subtracted from 1:

\begin{align}
    S_\text{hyp}(n, H, s, h) = 1 - \text{CDF}_\text{hyp}(n, H, s, h)
\end{align}

Where for clarity, the hypergeometric probability mass function is:

\begin{align}
    \text{PMF}_\text{hyp}(n, H, s, h) = \frac{\binom{h}{n}\binom{H-s}{h-n}}{\binom{H}{h}}
\end{align}

With binomial coefficients defined as:

\begin{align}
    \binom{h}{n} = \frac{h!}{n!(h-n)!}
\end{align}

And, conventionally, the cumulative distribution function is:

\begin{align}
    \text{CDF}_\text{hyp}(n, H, s, h) = \sum_{n_i \leq n} \text{PMF}_\text{hyp}(n_i, H, s, h)
\end{align}

So defined, we can see that the pan-genome collector's curve $C(h)$ is equivalent to $C_1(h)$, while the core genome collector's curve :
\begin{align}
    C(h) = C_1(h) \\
    \hat{C}(h) = C_h(h)
\end{align}

\section*{$k$-mer based collector's curves}

The definition of the collector's curve is agnostic to the unit of genomic sequence, so the calculation of a $k$-mer based curve is identical to the orthogroup based curve, excepting that $X$ will be the number of $k$-mers and $x$ will represent a $k$-mer, rather than an orthogroup.


\end{document}
This is never printed

It  1\footnote[1]{$x_{ij}$
could also be a continuous value} 